% !TEX TS-program = pdflatexmk

\documentclass[11pt]{article}
\title{\textbf{The \textsf{apatitlepage} package}}
\author{\textbf{Alan Munn}\\Department of Linguistics and Languages\\Michigan State University\\\texttt{\href{mailto:amunn@msu.edu}{amunn@msu.edu}}}
\date{Version 1.0\\October 10, 2017}
\usepackage[T1]{fontenc}
\usepackage[margin=1in]{geometry}
\usepackage{titling}
\usepackage[utf8]{inputenc}
\usepackage{array, booktabs, multicol, fancyhdr, xspace,tabularx}
\usepackage{enumitem}
\usepackage{fancyvrb,listings,url}
\usepackage[sf,compact]{titlesec}
\usepackage{screenplay-pkg}
\usepackage[colorlinks=true]{hyperref}


\DefineShortVerb{\|}
\newcommand*\bs{\textbackslash}

\IfFileExists{luximono.sty}%
{%
  \usepackage[scaled]{luximono}%
}
{%
  \IfFileExists{beramono.sty}%
  {%
    \usepackage[scaled]{beramono}%
  }{}
}

  
\lstset{%
    basicstyle=\ttfamily\small,
    commentstyle=\itshape\ttfamily\small,
    showspaces=false,
    showstringspaces=false,
    breaklines=true,
    breakautoindent=true,
    captionpos=t
    language=TeX
}
  
\newcommand*{\pkg}[1]{\texttt{#1}\xspace}
\setitemize[1]{label={}}
\setitemize[2]{label={}}
\setdescription{font={\normalfont}}
\setlength{\droptitle}{-1in}

\lhead{}
\chead{}
\rhead{}
\lfoot{\emph{}}
\cfoot{\thepage}
\rfoot{}
\renewcommand{\headrulewidth}{0pt}
\renewcommand{\footrulewidth}{0pt}
\pagestyle{fancy}


\begin{document}
\maketitle
\thispagestyle{empty}
\renewcommand{\abstractname}{\sffamily Abstract}

\abstract{\noindent\begin{quote}This package provides a title page based on the APA journal style of the  \pkg{apa6} document class. This allows an APA formatted title page to be used in another document class.
\end{quote}
\section{Introduction}
This package arose out of a question asked on the StackExchange website: \href{https://tex.stackexchange.com/q/395309/}{Manually create APA journal cover page} The question asked how easy it would be convert use class title page functionality in another document.  This package is the result of that discussion.
\section{Package use}
To use the package just load it like any other package:

\begin{lstlisting}
    \usepackage{apatitlepage}
\end{lstlisting}

\subsection{New commands}
The package renames some of the \pkg{apa6} commands to avoid name clashes and adds three formatting hooks for the title, authors and affiliations in case those need to be changed.

All other commands for entering authors and affiliations are identical to those described in the \href{https://www.ctan.org/pkg/apa6}{\pkg{apa6}} class. Please consult its  documentation for details.

\begin{center}

\begin{tabular}{l>{\raggedright\arraybackslash}p{3.5in}}
\toprule
\textsf{Command}       & \textsf{Explantion}\\
   |\APAtitle{}| 	   & replaces |\title{}|\\
   |\APAauthor{}|      & replaces |\author{}|\\
   |\APAshorttitle|    & replaces  |\shorttitle|\\
   |\APAmaketitle|	   & replaces |\maketitle|\\
   |\APAabstract{}|      & replaces |\abstract|\\
   |\APAkeywords{}|      & replaces |\keywords|\\
   |\APAauthornote{}|    & replaces |\authornote|\\
   |\APAnote{}|          & replaces |\note|\\
   |\APAtitleformat|   & defaults to |\LARGE|\\
       					 & use |\renewcommand| to define as needed\\
   |\APAauthorformat|  & defaults to |\large|\\
       					 & use |\renewcommand| to define as needed\\
   |\APAaffilformat|  & defaults to |{}|\\
       					 & use |\renewcommand| to define as needed\\			 
\bottomrule
\end{tabular}
\end{center}
\clearpage
\subsection{Unchanged commands}
All other titling commands relating to authors and affiliations have been retained.

\subsection{Eliminated commands}
Other than the titling commands, all other commands from the \pkg{apa6} class have been removed. There is also no anonymization code included in the package.


\subsection{Bugs}
You are welcome to report bugs and submit feature requests, but I should warn you that this package is extremely low priority for me in terms of maintenance, as I do not use it at all.  If you are interested in taking it over, please get in touch with me.

\section{Sample document}
The following is a sample document showing how the package is used. It can be found in the documentation folder for the package.
\lstinputlisting{apatitlepage-test.tex}

\end{document}
